\subsection{内存管理}

\subsubsection{寻址(\emph{Addressing})}
要理解内存寻址,
首先要理清三个地址的含义和它们的关系。
硬件和软件配合完成地址之间映射的过程。
\begin{description}
\item[物理地址] \hfill\\
  CPU将物理地址转为电信号,
  通过地址管脚发往地址总线,
  藉此获取或存入内存芯片上相应物理位置上的数据。
  在总线和每一片内存芯片之间有一个称为{\em memroy arbiter}
  的电路用于串行化所有的内存访问。
\item[逻辑地址] \hfill\\
  程序指令中使用的地址,
  不管内核还是用户态,
  这个地址值都要经过映射才能成为物理地址。
\item[线性地址] \hfill\\
  由于Intel平台需要同时支持{\em 段式寻址}和{\em 页式寻址},
  逻辑地址并不直接被映射为物理地址,
  而是先通过段式寻址得到一个中间状态,
  称为线性地址,
  之后再通过页式寻址得到最终的物理地址。
\end{description}
实模式的寻址比较简单,
只有段式寻址。
即逻辑地址值16位作为偏移地址,
段寄存器16位为基地址,
将其左移4位加上16位的偏移地址得到一个20位的物理地址,
也就是说实模式下只能寻址1M的地址空间。
另外要注意的是起始的640K由BIOS占用。

Intel保护模式的段式寻址稍微复杂。
每个段由某个段描述子定义,
共有4类段描述子:
描述代码段的Code Segment、
描述数据段的Data Segment、
Task State Segment、
及LDT Segment用于描述LDT地址段。
段描述子是64字节长的结构,
所有段描述子都保存于某张描述子表{\em gdt}或{\em ldt}中。
Gdt由内核维护并被所有进程共享,
用户进程需要操作段时可以通过它自己的ldt。
CPU 寻址过程见图~\ref{fig:laddr}

\begin{figure}[!ht]
  \centering
  \includegraphics[scale=0.8]{fig/laddr.pdf}
  \caption{Translating Logical Address}\label{fig:laddr}
\end{figure}

保护模式下逻辑地址值分为两个部分:
指令中使用的32位地址用于表示段内偏移。
某个16位段寄存器值用作某个段的{\em 段标识}({\em Segment Identifier}),
X86的段寄存器包括代码段寄存器\reg{cs}、
数据段寄存器\reg{ds},
栈段寄存器\reg{ss},
\reg{es}、\reg{fs},和\reg{gs}。
段标识的最低两位是试图访问该地址的CPU当前所拥有的权限,
总共可以有四个权限级别,
但Linux只使用其中的两个,
最高权限0和最低权限3。
段选择子第2位称为\verb|TI|({\em Table Indicator}),
用于指示这个地址做段式寻址使用的是gdt还是ldt。
段标识的高13位称为{\em 段选择子}({\em Segment Selector}),
就是在对应描述子表中的索引,
该描述子中的Base位域就是寻址的基地址部分。
而描述子中的DPL位域指明访问该地址所需要的权限。
所以可想而知,
段寄存器是进程上下文的一部分,
陷入内核后段寄存器会换成内核的段标识。

为加速寻址,
段描述子会被CPU缓存在某个不可编程的寄存器中,
不用每次都去内存读描述子表。

Intel引入段式寻址为了鼓励程序员将应用程序分割成不同的地址区域%
用于不同用途,
但Linux对段式寻址的支持非常有限,
Gdt中四个普通的段描述子,
\verb|__USER_CS|、
\verb|__USER_DS|、
\verb|__KERNEL_CS|、
及\verb|__KERNEL_DS|。
它们的值也几乎相同,
基地址全部是0,
换句话说就是段式地址是平的,
事实上没有分段。
权限位域用户空间为3,
内核空间为0.
另有几个有特殊用途的段;
\begin{enumerate*}[label=\itshape\alph*\upshape)]
\item Task State Segment,TSS是内核数据段中的一个小段,
  对应\verb|init_tss|数组,
  一个CPU id对应数组中对应位元素,
  每次发生内核陷入时,
  CPU从对应TSS读取内核栈地址然后完成上下文切换;
\item Thread-Local Storage Segment,
  用于存储线程私有的数据,
  线程可以调用\func{set\_thread\_area}和\func{get\_thread\_area}
  操作TLS,
  一个线程最多可以三个TLS;
\item Advanced Power Management;
\item Plug and Play;
\item 一个特别的TSS用于double fault。
\end{enumerate*}
Linux基本不使用ldt。

操作系统内核的一大职责就是在启动过程中填充各种数据结构
设定好gdtr,
为自己和之后的第一个用户进程配置段式寻址相关的寄存器。

Linux段式寻址是平的,
也就是说逻辑地址的值和线性地址相等。
当CPU的\emph{MMU}完成段式地址映射之后,
继续进行\emph{页式寻址},
最终得到用于读取或写入数据的物理地址。
页式寻址思路就是将线性内存空间划分为固定大小的区域,
区域内连续的地址被映射到连续的物理地址上,
相邻区域在物理地址空间上却不必须相邻。
这样的区域就是页,
在RAM上按页划分的区块就叫\emph{页帧}(\emph{Page Frame})。

MMU的页式内存管理部分负责页式寻址,
线性地址到物理地址的映射规则使用一个表(\emph{页表},\emph{Page Table})来描述,
表结构和格式是CPU和操作系统之间的一个约定,
操作系统内核通过填充页表来指示CPU完成页式寻址。

例如在X86平台上使用4K页帧,
最高20位指示每个物理页的起始地址。
一个平坦无层次结构的映射表总共包含$2^{20} = 1\text{M}$项,
每一项4字节,
于是一个进程的页映射表总共需要4M内存。
为节省空间,
Intel规定页表采用分层结构。
页式映射首先要通过cr3寄存器获得页目录地址,
线性地址最高10位用作页目录内索引,
一个页目录包含$2^{10} = 1\text{K}$项,
每一项保存下一层、
即页表某部分的起始地址,
一个页帧恰好可以用作\emph{页目录}作为进程页映射的起始点,
也就是说一个页目录项可以定位$2^{10} = 1\text{K}$个页表,
页目录在进程创建的时候分配并初始化,
其物理地址存入\reg{cr3}中成为进程上下文的一部分;
线性地址中间的10位用作页表内索引,
于是一个目录项指向的部分页表可以定位$2^{10}=1\text{K}$个物理页,
页表在进程运行期间动态分配;
32位线性地址的最低12位定位0到4095的页内偏移。
见图~\ref{fig:page}。
于是在x86平台上一个进程可以映射的物理内存尺寸为$1\text{K}\times 1\text{K} \times 4\text{K}=4\text{G}$。
如果寻址更多的物理内存,
就需要启用Intel提供的\emph{PAE}功能。

\begin{figure}[!ht]
  \centering
  \includegraphics[width=0.85\textwidth]{fig/page.pdf}
  \caption{Paging by x86 processor}\label{fig:page}
\end{figure}

页目录项和页表项的结构完全相同,
主要的有20位表示对应页表或页的起始地址,
另外的12位是页管理标志位。
\begin{itemize}
\item[\emph{Present Flag}:]
  如果MMU的页式映射单元在做寻址时发现某页目录项或页表项这一位被清除,
  它将要寻址的线性地址存入\reg{cr2},
  然后触发一个\emph{page fault}中断;
\item[\emph{ Accessed Flag}:]
  每一次MMU做寻址,
  访问到对应页目录项和页表项,
  会将它们的这这一标志置上,
  但MMU从来不清除该位。
  操作系统利用这一标志位实现按需分配页帧算法;
\item[\emph{ Dirty Flag}:]
  仅用于页表项,
  每次对页帧的写入都会置该标志位,
  内核可以决定根据这个标志决定是否将一个页刷到磁盘上;
\item[\emph{ Read/Write Flag}:]
  该页或页表的读写权限
\item[\emph{ User/Supervisor Flag}:]
  访问该页表或页所需的权限;
\item[\emph{ PCD and PWD Flag}:]
  决定硬件缓存如何处置页表和页帧;
\item[\emph{ Page Size Flag}:]
  决定页帧大小;
\item[\emph{ Global Flag}:]
  设置该标志可以使得该页的映射不从\emph{TLB}中清除。
\end{itemize}

x86支持4M大小的页帧,
\reg{cr4}寄存器的\reg{PSE}标志位表示系统将采用4M页帧。
当采用4M页帧时,
线性地址只被分为两个部分,
10位的页表索引和22位的页内偏移。

{\em PAE}的引入使得在32位硬件上可以访问大于4G的物理内存。
PAE机制就是增加一层寻址表,
{\em Page Directory Pointer Table,PDPT},
这张表中有4个64位的项,
\reg{cr3}指向的不再是页目录而是PDPT,
但对每个进程而言,
\reg{cr3}不再是固定的,
\reg{cr3}的值动态地指向PDPT中的某一项,
也就相当于切换整个页表空间。

64位系统线性地址如果划分为三部分,
页表仍然太大,
所以划分为4或5个部分,
4K页大小时,
x86\_64平台按$9 + 9 + 9 + 9 + 12$划分线性地址。

硬件缓存({\em hardware cache}),
在主存和缓存之间传输的最小单位不是字节,
而是缓存线,
x86\_64平台上缓存线尺寸位64字节。
可以有两种将主存映射到处理器缓存的思路,
一种称为{\em direct mapped},
某一地址总是映射到同一条缓存线上,
或者是称为{\em fully associative}的方式任意映射。
现实中一般采用{\em N-way set associative}的方式,
一个地址被映射到N条缓存线中的一条。
缓存命中时,
CPU上cache controller单元对读和写操作做不同处理。
读操作显然就是直接将缓存中内容返回,
不涉及主存操作;
写操作有两种策略,
{\em write through}直接写回主存,
也就是说压根就没有写缓存的概念,
而{\em write back}则不立即写回主存,
将修改的内容暂留在缓存线中,
之后当执行到某个需要刷新缓存线的指令(例如出现了cache miss)再将内容写回主存。
如果有相同主存被缓存在多CPU的,
所有CPU的缓存都将失效。
